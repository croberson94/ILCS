\documentclass[12pt,a4paper]{report}
\usepackage[latin1]{inputenc}
\usepackage[margin=1in,left=1.25in]{geometry}
\usepackage{amsmath}
\usepackage{amsfonts}
\usepackage{amssymb}
\usepackage{blindtext}
\usepackage{graphicx}
\usepackage{float}
\usepackage{indentfirst}
\usepackage{url}
\usepackage{subcaption}
\renewcommand{\thesection}{\arabic{section}}
\renewcommand{\figurename}{Fig.}



\begin{document}
\begin{titlepage}

	\centering
	\includegraphics[width=0.3\textwidth]{imgs/uttsealgray.png}\par\vspace{0.1cm}
	{\scshape\LARGE The University of Texas at Tyler\par}
	\vspace{0.1cm}
	{\scshape\LARGE College of Engineering\par}
	\vspace{1cm}
	{\scshape\Large EENG 4315 - Senior Design II\par}
	\vspace{1cm}
	{\huge\bfseries Intelligent Lighting Control System\par}
	\vspace{2cm}
	\textsc{a circadian based lighting system for space flight}\\
	\vspace{1cm}
	\textsc{Griggs Hutaff},  \textit{Co-Leader}\\
	\textsc{Charles Roberson},  \textit{Co-Leader}\\
	\textsc{Chad Dawili},  \textit{Financial Officer}\\
	\textsc{Ezequiel Fuentes},  \textit{Archivist}\\
	\textsc{Oscar Suram},  \textit{Acquisition Manager}
	\vfill

% Bottom of the page
	{\large \today\par}
\end{titlepage}
%\section*{Executive Summary}
%\thispagestyle{empty}
%\blindtext
\newpage
\renewcommand{\thepage}{\roman{page}}
\tableofcontents
\listoffigures
\listoftables
\newpage
\renewcommand{\thepage}{\arabic{page}}
\setcounter{page}{1}
\section{Project Description}
The goal of this project is to provide a lighting system which can meet the demands and aid the progression of long-term space flight. Although engineers have been able to overcome the immediate dangers of short range space flight we must further develop novel solutions to the issues of long-term confinement in artificial environments. Along with water and food, sleep is among the basic necessities for long term human survival. However, we know that astronauts suffer from sleep deprivation during flights. About half of everyone who flies to space relies on sleep medication and astronauts generally get about 6 hours of sleep in orbit despite being allowed 8.5.\cite{spacesleep}\\ \linebreak 
For this reason, NASA developed the "Lighting System to Improve Circadian Rhythm Control" to be used on the International Space Station (ISS). \cite{nasapatent} This modular lighting assembly uses a micro controller with power relay to adjust color temperature and perceived intensity. Future spacecrafts will require new and innovative light control methods to improve reliability such as compensating for degrading lighting sources and maintaining the crew's circadian rhythms \cite{tsgc}. \\ \linebreak
Our Intelligent Lighting Control System, centrally controlled with sensor feedback and visual status display, is a complete solution for future astronauts and their needs. Our system features an automatic light compensation algorithm, single communication bus capable of addressing each light fixture, and touchscreen user interface for customized sleep cycles. 



\section{Final Design Specifications}

%\begin{table}
%	\centering
%	\begin{tabular}{|c|c|}
%	\hline 
%	This is & the first table \\ 
%	\hline 
%	the first table & this is \\ 
%	\hline 
%	\end{tabular}
%	\caption{The First Table} 
%\end{table}
The Intelligent Lighting Control System is comprised of two interconnected parts, the control system and lighting modules. Our control system includes the Arduino UNO R3 for analog/digital input and ouptut, 4DUINO development board for touchscreen graphical user interface (GUI), AC to DC conversion power supply, RGB light and temperature sensors. Each light module include 3 RGB LED's, aluminum heat shield, Infineon RGB driver, and 3D printed housing for all parts.\\ \linebreak
Our control system features a light compensation algorithm which accounts for light degradation. The main issue identified by NASA engineers is light degradation due to yellowing of the light covers. Our sensors will measure the amount of red, green, and blue light spectrum generated from our light fixtures. If at any time the light spectrum emitted does not match the light spectrum measured the algorithm will begin adjusting the output of the light driver until the spectrum is back to normal.\\ \linebreak
The control system also allows the user to input a custom circadian-based cycle on a touchscreen interface. The interface allows central control of all light fixtures so that each light can be set to a different cycle to allow for shift work. The I2C (pronounced "I squared C") communication protocol allows us to control individual devices on a single bus which reduces the amount of cabling needed in the system.\\ \linebreak
The lighting modules have a two-piece modular design. The top piece can be permanently fixed to a ceiling or wall. The bottom piece which contains the LED's is screwed into the top piece with a threaded pattern on the outside which easily allows crew to replace LED's which have failed during flight. Each light module is equipped with heat shield and temperature sensor. In the event of overheating, the control system will trigger alarms to alert the crew of the issue.



\section{Design Solution}
%%%%%%%%%%%%%%%%%%%%%%%%%%%%%%%%%%%%%%%%%%%%%%%%%%%%%%%%%%%%%%%%%%%%%%
% THE BLOCK BELOW DEMONSTRATES HOW TO INSERT A FIGURE
% Simply place the .png file in the same directory as the .tex file
% and replace the file name between the " "
%%%%%%%%%%%%%%%%%%%%%%%%%%%%%%%%%%%%%%%%%%%%%%%%%%%%%%%%%%%%%%%%%%%%%% 

\begin{figure}[H]
	\centering
	\includegraphics[width=0.9\textwidth]{"imgs/High Level Design Overview gray".png}\par			\vspace{0.1cm}
	\caption{High Level System Overview}
\end{figure}
\subsection{Lighting Modules}
We set out to design a modular light fixture with interchangeable easy to replace parts. The lighting modules use Red-Blue-Green (RGB) light emitting diodes to produce a wide spectrum of light. Each lighting module features an independent current driver board with microprocessor which can communicate with our central microprocessor unit. The light fixture also features an interior aluminum heat shield and temperature sensor to manage excess heat and alert the control system of dangerous temperature levels. 
 
\subsubsection{RGB Light Emitting Diodes}
Light Emitting Diodes (LED) have many advantages over filament or gas based lights. LEDs are cheaper, lighter, last longer, and dissipate less heat. These advantages make them ideal for space flight. One disadvantage of LEDs is that they are not linear devices and making them behave in a linear fashion in regards to light intensity and color spectrum is not a trivial matter. The goal of a circadian based lighting system is to not only control the intensity of light but also the amount of red and blue light spectrum to simulate daily solar cycle on earth. \\ \linebreak 
To achieve these results we had to choose a solid state driver capable of controlling current in separate individual color channels. To insure rapid development we chose the Infineon RGB Lighting Shield as our LED driver. The heart of this boards functionality is the current controller using buck topology. 

\begin{figure}[H]
	\centering
	\includegraphics[width=0.75\textwidth]{"imgs/Buck".png}\par			
	\vspace{0.1cm}
	\caption{Buck Converter Current Controller}
	\label{fig:Buck}
\end{figure}

In this configuration, see Fig.~\ref{fig:Buck}, the inductor is constantly charging and discharging inside the circuit. After proper configuration the result is a constant current which can be changed to give us various intensities on a linear scale. We configured an acceptable ripple current (see I in figure 3.) which will result in constant light intensity.  The higher the ratio between input voltage and forward voltage, the more difficult it will be to configure a small ripple [4]. We can lower the input/forward voltage ratio by using a voltage divider to lower the input voltage. 

\subsubsection{RGB LED Driver}
Discuss the parts of the development board, the buck converter, the current limiter

\subsubsection{Light Fixture Housing}
The following represents the model for the light fixtures (a total count of 4). This design is intended to highlight the project as whole and to house the important components such the LED bulbs and controller. The plate is to act as a mount for the LED bulbs. The cutoff was designed as a pathway for the wires connecting to the shield. The middle is designed to hold the plate, LED bulbs and the heat sink. A lip was created in the bottom to catch and hold the LED plate in place. There is a 10 mm wide thread that will be used to screw onto the top housing which in then will hold the microprocessor.
\begin{figure}[H]
\centering
\begin{subfigure}{.5\textwidth}
  \centering
  \includegraphics[width=1\linewidth]{"display/Part1revised".png}
  \caption{Solid model}
  \label{fig:sub1}
\end{subfigure}%
\begin{subfigure}{.5\textwidth}
  \centering
  \includegraphics[width=1\linewidth]{"display/Part1revisedwire".png}
  \caption{Wireframe with hidden edges}
  \label{fig:sub2}
\end{subfigure}
\caption{3D CAD Model for LED plate}
\label{fig:test}
\end{figure}
\begin{figure}[H]
	\centering
	\includegraphics[width=.50\textwidth]{"display/ledplate".png}\par			
	\vspace{0.1cm}
	\caption{3D printed LED plate}
\end{figure}
\begin{figure}[H]
\centering
\begin{subfigure}{.5\textwidth}
  \centering
  \includegraphics[width=1\linewidth]{"display/Part2revised".png}
  \caption{Solid model}
  \label{fig:sub1}
\end{subfigure}%
\begin{subfigure}{.5\textwidth}
  \centering
  \includegraphics[width=1\linewidth]{"display/Part2revisedwire".png}
  \caption{Wireframe with hidden edges}
  \label{fig:sub2}
\end{subfigure}
\caption{3D CAD Model for middle housing}
\label{fig:test}
\end{figure}
\begin{figure}[H]
	\centering
	\includegraphics[width=.50\textwidth]{"display/middle".png}\par			
	\vspace{0.1cm}
	\caption{3D printed middle housing}
\end{figure}
\begin{figure}[H]
\centering
\begin{subfigure}{.5\textwidth}
  \centering
  \includegraphics[width=1\linewidth]{"display/Part3revised".png}
  \caption{Solid model}
  \label{fig:sub1}
\end{subfigure}%
\begin{subfigure}{.5\textwidth}
  \centering
  \includegraphics[width=1\linewidth]{"display/Part3revisedwire".png}
  \caption{Wireframe with hidden edges}
  \label{fig:sub2}
\end{subfigure}
\caption{3D CAD Model for top housing}
\label{fig:test}

\end{figure}
\newpage


\subsection{Control System}
\blindtext
\subsubsection{Central Microprocessor Unit}
Discuss Arduino, code, I2C protocol, Light Degradation Algorithm

\subsubsection{Graphical User Interface}
The graphical user interface we used for the Intelligent Lighting Control System is the 4Duino, an Arduino compatible display with built in 240x320 resolution TFT LCD Display with Resistive Touche and Wi-Fi capabilities. The display requires a uSD card to load required images for the program. When connected to our power supply, the display resets its communications and intiializes its setup routine by connecting to wireless communications. In the scenario any errors occur, the Callback Error Handler function raises flags for errors associated with setup. The 4Duino then mounts the uSD Card images and loads the program file, and connects to the Arduino slave on the I2C bus. The desired touch-screen interface objects are displayed after setup and can be interacted with after short delay. The GMT is generated with a modified NTP\_Clock routine that sends signals to the Arduino Slave indicating the time of day. The User Profile setting button allows the user to customize the GMT standard time with offsets. Warnings that are received from I2C communications with the Arduino slave are displayed to a webpage.

\subsubsection{Sensors}
Here we will discuss the temperature and light sensors

\subsubsection{Power Supply}
Due to NASA specifications, we chose an AC/DC power supply. The DC power will supply to the RGB LEDs, GUI, and Programmable Logic Controller. After performing our calculations on our simulated RGB circuit, we realized that necessary current and not voltage would be the greatest design concern. Our new choice for a power supply is a 24-volt DC, we changed the power supply again so that we have a slight change in current and voltage. The power supply that he had before had no case and just came with the printed circuit board, we felt that when powering the power supply there would be no protection for our teammates and the system. We also made a small circuit to step down the voltage from 24 volts to 5 volts, that voltage will be used to power the GUI and Programmable Logic Controller. 

\subsection{Display Model}
Our original design model was to 3D print a display piece inspired by the Orion spacecraft by NASA. However, we were unable to locate reasonably priced 3D printing services that could accommodate the overall dimensions of our model schematic. A commercial printer, which would be appropriately sized for our model, is not an option as the cost is not within our budget. It was determined as a group that doing so would not be advisable due to time constraints or possible future errors if broken down in parts. As a result, it has been decided to have wood, which would be painted white, as the material used for the display model which will be cost-effective towards our budget.
\begin{figure}[H]
	\centering
	\includegraphics[width=.75\textwidth]{"display/display model".png}\par			
	\vspace{0.1cm}
	\caption{3D CAD Display Model}
\end{figure}
\begin{figure}[H]
	\centering
	\includegraphics[width=.50\textwidth]{"display/model".png}\par			
	\vspace{0.1cm}
	\caption{Display Model}
\end{figure}

We determined that having an actual 3D printed light fixture would better suit our project, ultimately showcasing as one whole product. We discovered West Houston Institute's 3D printing services with their requirements being that the user is to provide their own materials such 3D filament, and to be a student currently enrolled to Houston Community College. We decided to utilize their services as this was the most cost-effective out of the services we had reached out to, and one of our members, Ezequiel, is a current HCC student.\\ \linebreak
The process to 3D printing was a bit of a task as none of us were experienced with both drafting a model and the service itself. First, we used Inventor to draft our model and convert the files into .STL, as these are one of the accepted files for 3D-printing use. Second, we chose the Ultimaker 3 3D printer and uploaded our files into its print software, Cura. Essentially that's all Cura is, a way to get a digital file from your computer to the 3D printer in a format that the 3D printing hardware understands.\\ \linebreak
This is when we ran into some complications. Our uploaded files were scaled down to 10 percent of its original size. We discovered that when converting our drawings into .STL files, we had to indicate and fix the settings with the units set to millimeters, not centimeters. Another obstacle was to pinpoint the fitting and sizes for the printouts as the final product tend to shrink when cooled down. After a few iterations, we have our fixtures printed.\\ \linebreak 

\newpage

\subsection{System Schematics}
\begin{figure}[H]
	\centering
	\includegraphics[width=1.2\textwidth, angle = 90]{"imgs/Arduino".png}\par			\vspace{0.1cm}
	\caption{Central Microprocessor Unit}
\end{figure}
\newpage
\begin{figure}[H]
	\centering
	\includegraphics[width=1.2\textwidth, angle = 90]{"imgs/4duino".png}\par			\vspace{0.1cm}
	\caption{Graphical User Interface (Input)}
\end{figure}
\newpage
\begin{figure}[H]
	\centering
	\includegraphics[width=1.2\textwidth, angle = 90]{"imgs/Infineon".png}\par			\vspace{0.1cm}
	\caption{Light Module (Single Unit)}
\end{figure}
\newpage
\begin{figure}[H]
	\centering
	\includegraphics[width=1.2\textwidth, angle = 90]{"imgs/LightSensor".png}\par			\vspace{0.1cm}
	\caption{Light Sensor Network}
\end{figure}
\newpage
\begin{figure}[H]
	\centering
	\includegraphics[width=1.2\textwidth, angle = 90]{"imgs/PowerSupply".png}\par			\vspace{0.1cm}
	\caption{Power Supply and DC Power Circuit}
\end{figure}
\newpage



%%%%%%%%%%%%%%%%%%%%%%%%%%%%%%%
%   DUE FEBRUARY 22nd!
%%%%%%%%%%%%%%%%%%%%%%%%%%%%%%%








%\section{Prototype Design and Fabrication}
%\blindtext
%\section{Testing and Validation}
%\blindtext
%\section{Manufacturing Methods}
%\blindtext
%\section{Broader Impacts of the Project}
%\blindtext
%\section{Conclusions}
%\blindtext
\setcounter{section}{8}
\section{References}
\begingroup
   \def\chapter*#1{}
   \bibliography{SeniorDesign}
	
\endgroup	

 

%\section{Appendices}
%\blindtext
%%Include materials that could not be included in the main body of the report without disrupting its continuity or extending its length to unreasonable limits.  List all of the appendices by title in the table of contents.  
%
%%The first page of each appendix is numbered <appendix letter>1, (e.g., 
%
%\subsection{Appendix A: Test Protocols}
%Include written test protocols used for all testing.  The protocols should be given descriptive names. The protocols should include descriptions of how the testing is to be performed and what data are to be collected, and should include sample data collection sheets.\\
%If the "testing" is to be performed with computer simulation, test protocols must describe such things as a description of the model used in the simulation, the parameters to be varied and their ranges, and the boundary or other environmental conditions applied.\\
%
%
%\subsection{Appendix B: Test Results}
%Summarize, tabulate, or chart the results of all testing. Results must reference the appropriate test protocols.  All test results must include the name(s) of the person(s) responsible for conducting the tests.\\
%
%\subsection{Appendix B: Codes, Standards, Constraints}
%Use the same table as included in the PDD but now cite the specific Codes and Standards that applied by name and number and state in which sections of the report they were applied.  The blank table is reproduced below.  An entry is required in Table C for each consideration or constraint.  Justifications must be provided for any constraint that is judged to be not applicable.\\
%
%\subsection{Appendix (-ices) D -} 
%Include materials that could not be included in the main body of the report without disrupting its continuity or extending its length to unreasonable limits.  Do not copy manufacturers' entire datasheets to appendices.  



\end{document}