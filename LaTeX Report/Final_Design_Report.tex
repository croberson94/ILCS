\documentclass[12pt,a4paper]{report}
\usepackage[latin1]{inputenc}
\usepackage[margin=1in,left=1.25in]{geometry}
\usepackage{amsmath}
\usepackage{amsfonts}
\usepackage{amssymb}
\usepackage{blindtext}
\usepackage{graphicx}
\usepackage{float}
\usepackage{indentfirst}
%\usepackage{biblatex}
%\addbibresource{SeniorDesign.bib}
\renewcommand{\thesection}{\arabic{section}}
\renewcommand{\figurename}{Fig.}

%\addbibresource{SeniorDesign.bib}

\begin{document}
\begin{titlepage}

	\centering
	\includegraphics[width=0.3\textwidth]{uttsealgray.png}\par\vspace{0.1cm}
	{\scshape\LARGE The University of Texas at Tyler\par}
	\vspace{0.1cm}
	{\scshape\LARGE College of Engineering\par}
	\vspace{1cm}
	{\scshape\Large EENG 4315 - Senior Design II\par}
	\vspace{1cm}
	{\huge\bfseries Intelligent Lighting Control System\par}
	\vspace{2cm}
	\textsc{a circadian based lighting system for space flight}\\
	\vspace{1cm}
	\textsc{Griggs Hutaff},  \textit{Co-Leader}\\
	\textsc{Charles Roberson},  \textit{Co-Leader}\\
	\textsc{Chad Dawili},  \textit{Financial Officer}\\
	\textsc{Ezequiel Fuentes},  \textit{Archivist}\\
	\textsc{Oscar Suram},  \textit{Acquisition Manager}
	\vfill

% Bottom of the page
	{\large \today\par}
\end{titlepage}
%\section*{Executive Summary}
%\thispagestyle{empty}
%\blindtext
\newpage
\renewcommand{\thepage}{\roman{page}}
\tableofcontents
\listoffigures
\listoftables
\newpage
\renewcommand{\thepage}{\arabic{page}}
\setcounter{page}{1}
\section{Project Description}
The goal of this project is to provide a lighting system which can meet the demands and aid the progression of long-term space fight. Although engineers have been able to overcome the immediate dangers of short range space flight we must further develop novel solutions to the issues of long-term confinement in artificial environments. Along with water and food, sleep is among the basic necessities for long term human survival. For this reason, NASA developed the Lighting System to Improve Circadian Rhythm Control to be used on the International Space Station (ISS). \cite{nasapatent} This modular lighting assembly uses a micro controller with power relay to adjust color temperature and perceived intensity.\\\linebreak
Future spacecrafts will require new and innovative light control methods to improve reliability such as compensating for degrading lighting sources and maintaining the crew's circadian rhythms \cite{tsgc}. Our Intelligent Lighting Control System, centrally controlled with sensor feedback and visual status display, is a complete solution for future astronauts and their needs. Our system features an automatic light compensation algorithm, single communication bus capable of addressing each light fixture, and touchscreen user interface for customized sleep cycles. 

%%%%%%%%%%%%%%%%%%%%%%%%%%%%%%%%%%%%%%%%%%%%%%%%%%%%%%%%%%%%%%%%%%%%%%
% THE BLOCK BELOW DEMONSTRATES HOW TO INSERT A FIGURE
% Simply place the .png file in the same directory as the .tex file
% and replace the file name between the " "
%%%%%%%%%%%%%%%%%%%%%%%%%%%%%%%%%%%%%%%%%%%%%%%%%%%%%%%%%%%%%%%%%%%%%% 

\begin{figure}[H]
	\centering
	\includegraphics[width=0.75\textwidth]{"High Level Design Overview gray".png}\par\vspace{0.1cm}
	\caption{High Level System Overview}
\end{figure}

\section{Final Design Specifications}

%\begin{table}
%	\centering
%	\begin{tabular}{|c|c|}
%	\hline 
%	This is & the first table \\ 
%	\hline 
%	the first table & this is \\ 
%	\hline 
%	\end{tabular}
%	\caption{The First Table} 
%\end{table}
The Intelligent Lighting Control System is comprised of two interconnected parts, the control system and lighting modules. Our control system includes the Arduino UNO R3 for analog/digital input and ouptut, 4DUINO development board for touchscreen graphical user interface (GUI), AC to DC conversion power supply, RGB light and temperature sensors. Each light module include 3 RGB LED's, aluminum heat shield, Infineon RGB driver, and 3D printed housing for all parts.\\ \linebreak
Our control system features a light compensation algorithm which accounts for light degradation. The main issue identified by NASA engineers is light degradation due to yellowing of the light covers. Our sensors will measure the amount of red, green, and blue light spectrum generated from our light fixtures. If at any time the light spectrum emitted does not match the light spectrum measured the algorithm will begin adjusting the output of the light driver until the spectrum is back to normal.\\ \linebreak
The control system also allows the user to input a custom circadian-based cycle on a touchscreen interface. The interface allows central control of all light fixtures so that each light can be set to a different cycle to allow for shift work. The I2C (pronounced "I squared C") communication protocol allows us to control individual devices on a single bus which reduces the amount of cabling needed in the system.\\ \linebreak
The lighting modules have a two-piece modular design. The top piece can be permanently fixed to a ceiling or wall. The bottom piece which contains the LED's is screwed into the top piece with a threaded pattern on the outside which easily allows crew to replace LED's which have failed during flight. Each light module is equipped with heat shield and temperature sensor. In the event of overheating, the control system will trigger alarms to alert the crew of the issue.



\section{Design Solution}
\blindtext
%%%%%%%%%%%%%%%%%%%%%%%%%%%%%%%
%   DUE FEBRUARY 22nd!
%%%%%%%%%%%%%%%%%%%%%%%%%%%%%%%








%\section{Prototype Design and Fabrication}
%\blindtext
%\section{Testing and Validation}
%\blindtext
%\section{Manufacturing Methods}
%\blindtext
%\section{Broader Impacts of the Project}
%\blindtext
%\section{Conclusions}
%\blindtext
\setcounter{section}{8}
\section{References}
\begingroup
   \def\chapter*#1{}
   \bibliography{SeniorDesign}
	\bibliographystyle{plain}
\endgroup	

 

%\section{Appendices}
%\blindtext
%%Include materials that could not be included in the main body of the report without disrupting its continuity or extending its length to unreasonable limits.  List all of the appendices by title in the table of contents.  
%
%%The first page of each appendix is numbered <appendix letter>1, (e.g., 
%
%\subsection{Appendix A: Test Protocols}
%Include written test protocols used for all testing.  The protocols should be given descriptive names. The protocols should include descriptions of how the testing is to be performed and what data are to be collected, and should include sample data collection sheets.\\
%If the "testing" is to be performed with computer simulation, test protocols must describe such things as a description of the model used in the simulation, the parameters to be varied and their ranges, and the boundary or other environmental conditions applied.\\
%
%
%\subsection{Appendix B: Test Results}
%Summarize, tabulate, or chart the results of all testing. Results must reference the appropriate test protocols.  All test results must include the name(s) of the person(s) responsible for conducting the tests.\\
%
%\subsection{Appendix B: Codes, Standards, Constraints}
%Use the same table as included in the PDD but now cite the specific Codes and Standards that applied by name and number and state in which sections of the report they were applied.  The blank table is reproduced below.  An entry is required in Table C for each consideration or constraint.  Justifications must be provided for any constraint that is judged to be not applicable.\\
%
%\subsection{Appendix (-ices) D -} 
%Include materials that could not be included in the main body of the report without disrupting its continuity or extending its length to unreasonable limits.  Do not copy manufacturers' entire datasheets to appendices.  



\end{document}